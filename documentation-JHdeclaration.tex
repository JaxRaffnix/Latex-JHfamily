% !TEX TS-program = pdflatex
% !TEX encoding = UTF-8 Unicode

\documentclass[%
	fontsize=10pt, 
	DIV=8, 
%	twocolumn
]{scrartcl}

\usepackage[T1]{fontenc}
\usepackage[utf8]{inputenc}
\usepackage[ngerman]{babel}
\usepackage{palatino}

\usepackage[dvipsnames]{xcolor}
\usepackage{listingsutf8}
\usepackage{booktabs}
\usepackage{hyperref}

\usepackage{JHdeclaration}

\definecolor{bluelinks}{rgb}{0.16, 0.32, 0.75}
\hypersetup{%
	bookmarksnumbered	=	true,			%	Überschriften in pdf-Reader nummerieren
	breaklinks			=	true,			%	Links umbrechen
	colorlinks			=	true,			%	farbliche Links
	urlcolor			=	bluelinks,
	linkcolor			=	.,
	filecolor			=	.,
	citecolor			=	.,
}

\newcommand{\commentcolor}{\color{ForestGreen}}
\newcommand{\stringcolor}{\color{Mulberry}}
\newcommand{\keywordcolor}{\color{blue}}
\lstset{%
	frame			=	tb ,							%	horizontale Linie oben&unten
	breaklines		=	true,							%	Zeilenumbruch
	rulecolor		=	\color{black} ,					%	Rahmenfarbe ist schwarz
	keywordstyle	=	\keywordcolor ,
	commentstyle	=	\commentcolor ,
	stringstyle		=	\stringcolor ,
	title			=	\lstname ,						%	Titel ist gleich dem Dateinamen
	basicstyle		=	\footnotesize\ttfamily ,		%	Kleine Schrift und Monospace
	numbers			=	left,							%	Zeilennumber rechts (da marginalie links)
	inputencoding	=	utf8,  							% Input encoding
    extendedchars	=	true,  							% Extended ASCII
}
\lstset{literate=%							%	ermöglicht Unicode-Zeichen in Listing!
    {Ö}{{\"O}}1
    {Ä}{{\"A}}1
    {Ü}{{\"U}}1
    {ß}{{\ss}}1
    {ü}{{\"u}}1
    {ä}{{\"a}}1
    {ö}{{\"o}}1
    {~}{{\textasciitilde}}1
	{*}{{\normalfont{*}}}1
}
\lstdefinestyle{TeX}{									%	mehr Keywörter wenn style=TeX
	morekeywords={vspace, hspace, rule, ifdefined, newcommand, setlength, newlength, RequirePackage, ProvidesPackage, NeedsTeXFormat, DeclareOption, ProcessOption, define, PackageWarning, CurrentOption, ProcessOptions, if, setkeys, define@boolkey, define@key, documentclass, usepackage, renewcommand, begin, familydefault, sfdefault, enquote,}
}
\renewcommand{\lstlistingname}{Code}						%	Snippet umbenennen
\renewcommand{\lstlistlistingname}{Codeverzeichnis}				%	Listoflistings umbenennen

\title{Das Paket \textit{JHdeclaration}}

\author{Jan Hoegen\\\href{mailto:jan.hoegen@web.de}{jan.hoegen@web.de}}

\date{28. April 2021}

\begin{document}

\maketitle

\section{Einführung}
Dieses Paket stellt an die gewünschte Stelle die Eidesstattliche Erklärung für Abschlussarbeiten des \textit{Karlsruher Institut für Technologie}. Der dafür notwendige Befehl lautet
\begin{verbatim}
	\declaration[<>]{<>}
\end{verbatim} 
Es wird automatisch von der Singular- in die Pluralform gewechselt, wenn mehrere Autoren angegeben sind.\par
Damit auch im \verb+twoside+ -Modus keine Fehler auftreten, muss diese Anweisung zwingend global beim laden der Dokumentenklasse
\begin{verbatim}
	\documentclass[twoside, <>]{<>}
\end{verbatim} 
angegeben werden.\par
Bugs und Verbesserungsvorschläge können an den Autor über \href{mailto:jan.hoegen@web.de}{jan.hoegen@web.de} gesendet werden.

\section{Laden des Pakets}

Um das Paket nutzen zu können muss die Datei \verb+JHdeclraration.sty+ entweder im gleichen Ordner wie das Hauptdokument oder im Installationspfad der \TeX -Distribution liegen. \par
Anschließend kann das Paket über
\begin{verbatim}
	\usepackage{JHdeclaration}
\end{verbatim}
eingebunden werden.

\section{Verwenden des Pakets}
Der Text mit Überschrift und Singnaturzeile wird mit dem Befehl 
\begin{verbatim}
	\delaration[<>]{<>}
\end{verbatim}
erzeugt. Im notwendigen Argument \verb+{}+ wird der Name des Autors angegeben.

Weitere Parameter können entweder mit Hilfe des Befehls
\begin{verbatim}
	\JHsetup{<>}
\end{verbatim}
oder als optionales Argument beim Aufrufen des \verb+\declaration+ Befehls angegeben werden.
	\subsection{Optionale Argumente}
	Folgende optionale Argumente sind möglich:
	\begin{description}
%		\item \verb+heading+ Legt die Gliederungsebene der Überschrift fest
%		\item \verb+numbering+ Nummerierung für die Überschrift ein- bzw. ausschalten
%		\item \verb+authortwo+ Vorname und Nachname des zweiten Autors festlegen
%		\item \verb+authorthree+ Vorname und Nachname des dritten Autors festlegen
		\item[blank] Es wird keine Überschrift und kein Text überzeugt, nur die Signaturzeile(n)
		\item[heading] Legt die Gliederungsebene der Überschrift fest
		\item[numbering] Nummerierung für die Überschrift ein- bzw. ausschalten
		\item[authortwo] Vorname und Nachname des zweiten Autors festlegen
		\item[authorthree] Vorname und Nachname des dritten Autors festlegen
	\end{description}
	
	Tabelle \ref{tab:1} zeigt die Standardwerte für die verschiedenen Elemente.
	
	\begin{table}
		\centering
		\begin{tabular}{lll}\toprule
			Element		&	Mögliche Werte	&	Standard\\\midrule
			blank		&	true, false		&	false\\
			heading		&	chapter, addchap, section, addsec, subsection	&	section\\
			numbering	&	true, false	&	true\\
			authortwo	&	<>		&	\\
			authorthree	&	<>		&	\\\bottomrule
		\end{tabular}
		\caption{Optionale Argumente von declaration}
		\label{tab:1}
	\end{table}		
	
\newpage
	
\section{Beispiel}
Und nun ein kleines Beispiel:
	
	\begin{center}
	\begin{minipage}[c]{0.48\textwidth}
		\begin{lstlisting}[language=TeX, style=TeX, caption={Minimalbeispiel}]
\documentclass[fontsize=10pt, DIV=8]{scrartcl}

\usepackage[T1]{fontenc}
\usepackage[utf8]{inputenc}
\usepackage[ngerman]{babel}
\usepackage{lmodern}
\renewcommand*\familydefault{\sfdefault}

\usepackage{JHdeclaration}

\begin{document}

\declaration[%
	heading=addsec,
	numbering=false,
	authortwo={Peter Lustig},
	authorthree={Max Müller},
]{Jan Hoegen}

\end{document}
		\end{lstlisting}
	\end{minipage}%
	\hfil%
	\begin{minipage}[c]{0.48\textwidth}
		\begingroup\fontfamily{lmss}\selectfont\small
		\declaration[%
			heading=addsec,
			numbering=false,
%			authortwo={Peter Lustig},
%			authorthree={Max Müller},
		]{Jan Hoegen}
		\endgroup
	\end{minipage}
	\end{center}

\section{Der Code}
\lstinputlisting[language=TeX, style=TeX]{JHdeclaration.sty}

\declaration[%
]{Jan Hoegen ü}

\end{document}